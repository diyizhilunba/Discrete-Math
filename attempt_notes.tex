\documentclass[a4paper]{article}

\def\npart {MA-2314}
\def\nterm {Spring}
\def\nyear {2024}
\def\nlecturer {Cereste Ken}
\def\ncourse {Discrete Math}

\input{header}

\begin{document}
\maketitle
\section{Definitions}
\subsection{Relation}
Let $A$ and $B$ be sets. A relation $R$ from $A$ to $B$ is a subset of $A \times B$. The set $A$ is called the domain of $R$ and the set $B$ is called its co-domain.
\subsection{Function}
A relation that satisfies the following (two) property:
\begin{enumerate}
    \item every x has one and only one corresponding y
    \item every x has only one corresponding y
\end{enumerate}
\begin{remark}
    A function has no restriction on y\\
    If it does not have such a property, we call this function not well-defined
\end{remark}

\subsection{Range, image, and Codomain}
Range = Image = All y that has a corresponding x\\
Codomain = all the possible y in the set(The entire set that y can select from)\\
The image of x under f = f of x = the value of f at x = $\{y \in Y\mid y = f(x) \text{for some } x \in X\}$

\subsection{Inverse image and premiage}
Inverse = preimage = X, as all x in X has to have a corresponding y

\subsection{Identity function}
$I_X(x) = x$ for all $x \in X$

\subsection{What it actually means for a function(How use in prove)}
$f: X \rightarrow Y$, let $A, B \subseteq X$\\
What $f(A)$ means:\\
\textbf{In y perspective}\\
Suppose(for any) $y \in f(A)$, there exists $x \in A$ such that $f(x) = y$\\
Here, we see $f(A)$ similar to what we said in image(A set of things that can be related to all $x \in A$\\
\textbf{In x perspective}\\
There exists a $y$ such that $y = f(x)$ for some $x \in A$, Therefore, $y \in f(A)$
\subsection{One to one and how to prove:}
If$f(x_1) = f(x_2)$, then $x_1 = x_2$\\
\textbf{Prove True:}\\
$\forall x_1, x_2 \in A, f(x_1) = f(x_2) \rightarrow x_1 = x_2$\\
Suppose (Condition),..., $x_1 = x_2$\\
\textbf{Prove False:}\\
$\exists x_1, x_2 \in A, f(x_1) = f(x_2) \vee x_1 \neq x_2$\\
\subsection{Onto and how to prove:}
$\forall y \in Y, \exists x \in X (F(x) = y)$\\
Suppose y be any element in Y, ... , there exists $x = \ldots$ such that $F(x) = y$ \\
\textbf{Prove not:} there exist one that does not have corresponding x\\
\subsection{Inverse Function}
Suppose \( F: X \rightarrow Y \) is a one-to-one correspondence; in other words, suppose \( F \) is one-to-one and onto. Then there is a function \( F^{-1}: Y \rightarrow X \) that is defined as follows: Given any element \( y \) in \( Y \),

\[ F^{-1}(y) = \text{that unique element } x \text{ in } X \text{ such that } F(x) \text{ equals } y. \]

Or, equivalently,

\[ F^{-1}(y) = x \iff y = F(x). \]
\subsection{String Equality}
$s_1,s_2$ are the same iff they have the same length and exactly the same character\\
Then we can remove/add a new string at the same position



\section{Prove log is irrational}
\begin{ex}
    Prove $log_37$ is irrational\\
\begin{proof}
    Assume $log_37$ is rational\\
    By definition of rational number, Let $log_37 = \frac{p}{q}$ where $p,q \in \Z, q \neq 0$\\
    By definition of $log$, $7 = 3^{\frac{p}{q}}$\\
    By algebra, $7^q = 3^p$\\
    Let $u = 7^q$\\
    As $\Z$ is closed under multiplication, $u \in \Z$\\
    By substitution, $u = 7^q = 3^p$\\
    By undue factorization Theorem, u can only have one way to be composed as a multiplication of primes\\
    Therefore there is a contradiction with the statement\\
    As a result, $log_37$ is rational
    
\end{proof}
\end{ex}

\section{Prove $f(A) \subseteq f(B)$}
\begin{ex}
    $F(A\cap B) \subseteq F(A) \cap F(B)$
\begin{proof}
    \textit{Define everything:}\\
    Let $F$ be a function from X to Y, \\
    Assume $A, B$ be arbitrary set in X\\
    Let $y \in F(A \cap B)$         \textit{$\#$We cannot say that y is arbitrary here?}\\
    By definition of the image of a set, there must be some x such that $x\in (A \cap B), f(x) = y$\\
    ... $x \in A, x \in B$\\
    Therefore $y = F(x)$ for some $x \in A, x \in B$\\
    Since $y = F(x)$ for some $x \in A$\\
    $y \in F(A)$ by definition of image of a set\\
    ... $y \in F(B)$ by definition of image of a set\\
    therefore $y \in F(A) \cap F(B)$ ...\\
    Therefore, $F(A \cap B) \subseteq F(A) \cap F(B)$\\
\end{proof}
\end{ex}





\begin{thm}
    If $f: X \rightarrow Y, g: Y \rightarrow Z $ are both injective functions, then $g \circ f$ is injective
\end{thm}
\begin{proof}
    Let f,g bothe be injective functions\\
    Let $x_1,x_2 \in X$, Suppose $(g\circ f)(x_1) = (g \circ f)(x_2)$\\
\begin{align*}
    (g\circ f)(x_1) &= (g \circ f) (x_2)\\
    g(f(x_1)) &= g(f(x_2))
    f(x_1) &= f(x_2)\\
    x_1 &= x_2
\end{align*}
\end{proof}

\begin{remark}
    Let $f: X \rightarrow Y, g: Y\rightarrow Z$ be any functions\\
    Suppose $g \circ f$ is injective
    Let $x_1, x_2 \in X, f(x_1) = f(x_2)$\\
    \begin{align*}
        f(x_1) &= f(x_2)\\
        g(f(x_1)) &= g(f(x_2))\\
        (g \circ f)(x_1) &= (g \circ f)(x_2) \\
    \end{align*}
    therefore, f must be injective\\
    However,
    \[
    \exists y_1,y_2 \in Y(g(y_1) = g(y_2) \wedge y_1 \neq y_2)
    \]
\end{remark}


\section{PigeonHole Theorem}
\textbf{A Set-Theoretic Notation:}\\
Let $Y^X$ denote the set of all functions from a set X into a set Y
\[
Y^X = \{f \subseteq X \times Y \mid f:X \rightarrow Y\}
\]
\textbf{The Pigeonhole Principle(PHP):}\\
A function from one finite set to a smaller finite set cannot be injective. There must be at least two elements in the domain that have the same image in the co-domain\\
or
\begin{thm}
    \textbf{PegionHole Theorem:}Let X,Y be sets such that $N(X) = n \in \Z^+, N(y) = m \in \Z^+$ For any $f \in Y^X$, if $n > m$, then f is not jnjective, i.e. there exist $x_1,x_2 \in X$such that $f(x_1) = f(x_2) , x_1 = x_2 $  
\end{thm}

\begin{thm}
    Let $X,Y$ be sets such that $N(x) = N(y) = n \in \Z^+ , f \in Y^X$ f is injective iff f is surjective
\end{thm}
\begin{exercise}
    Let $n\in \Z^+$ such that $S = \{1,2,\ldots,2n\}$\\
    If we choose $n+1$ integers from S, must at least one of them be even?\\
    If we choose $n+1$ integers from S, must at least one of them be odd?
\end{exercise}
\begin{proof}
    
\end{proof}


\end{document}