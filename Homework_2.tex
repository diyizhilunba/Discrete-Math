\documentclass{article}

\usepackage{amsmath}
\usepackage{amssymb}
\usepackage{array}
\usepackage{mathrsfs}

\title{Homework 2}
\author{Aaron Ma}
\begin{document}
\maketitle
\section{3.1}
\subsection{4}
\subsubsection{a}
If $x = -2$ and $y = -1$,
then $x < y$, meaning the hypothesis is true.\\
However, $x^2 = 4$ and $y^2 = 1$,
then $x^2 > y^2$, which contradicts the conclusion.\\
This states that $Q(x,y)$ has a true hypothesis but a false conclusion when $x = -2$ and $y = -1$. In other words, it is false\\
\subsubsection{b}
when $x = -3$, $y = 2$,
$x < y$ but $x^2 = 9 > y^2 = 4$\\
This indicates that $Q(-3,2)$ has a true hypothesis but a false conclusion.\\
That is to say that it is false.
\subsubsection{c}
If $x = 3$ and $y = 8$,
then $x < y$, meaning the hypothesis is true.\\
In addition, $x^2 = 9$ and $y^2 = 64$,
then $x^2 < y^2$, which means the conclusion is true\\
This states that $Q(x,y)$ has a true hypothesis and a true conclusion when $x = 3$ and $y = 8$. In other words, it is true\\
\subsubsection{d}
when $x = 6$, $y = 9$,
$x < y$ and $x^2 = 36 < y^2 = 81$\\
This indicates that $Q(6,9)$ has a true hypothesis and a true conclusion.\\
That is to say that it is true.\\
\subsection{20}
The square root of a positive number is real\\
Having a positive square root is necessary for a positive number\\
\subsection{32}
\subsubsection{b}
for any $x > 2$,\\
\begin{align*}
    x^2-2^2 = (x+2)(x-2)\\
    \intertext{As $x > 2$, $x+2 > 0$ and $x-2>0$}\\
    \therefore (x+2)(x-2) > 0
\end{align*}
As a result, $x^2 - 2^2 > 0$, i.e. $x^2 > 4$\\
The conclusion is true.
\subsubsection{d}
The conclusion is True
\section{3.2}
\subsection{15}
\subsubsection{b}
$\forall x \in D$ that is less than 0, $x \in {-48,-14,-8}$\\
As $-48,-14,-8$ are all even, $x$ must be even.\\
So the statement is true
\subsubsection{d} 
$\forall x \in D$ that is less than 0, $x \in {32}$\\
As the tens digit of 32 is 3, which is 3 or 4\\
So the statement is true
\subsubsection{e}
$\forall x \in D$ that the ones digit is 6, $x \in$ {16,26,36}\\
If $x$ equals 36, the tens digit is 3 but not 1 or 2. \\
So the statement is false
\section{3.2}
\subsection{12}
Statement:\\
The product of any irrational number and any rational number is irrational.\\
Its inverse statement is that there exists such a pair of one irrational number and one rational number that their product is rational.\\
Proposed negation is the product of any irrational number and any rational number is rational, which is different from the negation.
\subsection{40}
if a number is divisible by 8, then it is divisible by 4
\subsection{46}
All happy people have a large income
\section{3.3}
\subsection{43}
There exists a real number $\varepsilon > 0$ that there isn't any real number $\vartheta$  such that for every real number $x$, if $a-\vartheta < x < a+\vartheta $  and $x \neq a$ then $L - \varepsilon < f(x) < L + \varepsilon$
\subsection{44}
\subsubsection{a}
The statement is true\\
Existence:\\
if $x = 1$, $\forall y \in R, xy = y$\\
Uniqueness:\\
if $\exists z \neq 1(\forall y \in R, zy = y)$\\
$z = \frac{y}{y} = 1$ contradicts with the assumption\\
So only 1 can meet the requirement and the statement is true\\
\subsubsection{b}
The statement is false\\
if $x = 1, \frac{1}{x} = 1$, which is an integer\\
if $x = -1, \frac{1}{x} = -1$, which is also an integer\\
\subsubsection{c}
The statement is true\\
Choose $y = -x$, As $x$ is a real number, $y$ is also a real number\\
$\therefore x + y = 0$ and the statement is true\\
Also, note that if $x+y = 0$, then $y = -x$, meaning the selection of $y$ is unique
\subsection{45}
There exists one and only one $x \in D$, such that $P(x)$
\section{3.3}
\subsection{56}
They may have different truth value
Let us assume that $P(x)$ is $x$ is even, $Q(x)$ is $x$ is odd, and $D$ is all integer
As a result,
\begin{align*}
    (\exists x \in D, P(x)) \wedge  (\exists x \in D, Q(x))
    \intertext{Means there exists integer x that is even, and there also exists integer y that is odd, which is true}
    (\exists x \in D, P(x) \wedge Q(x))
    \intertext{Means there exists integer x that is both even and odd, which is false}
\end{align*}
\subsection{57}
They may have different truth value\\
Let us assume that $P(x)$ is $x$ is even, $Q(x)$ is $x$ is odd, and $D$ is all integer
As a result,
\begin{align*}
    (\forall x \in D, P(x)) \vee (\forall x \in D, Q(x))
    \intertext{Means all integers are even or all integers are odd, which is false}
    \forall x \in D, (P(x) \vee Q(x))
    \intertext{Means all integers are either odd or even, which is true}\\
\end{align*}



\end{document}