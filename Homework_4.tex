\documentclass{article}

\usepackage{amsmath}
\usepackage{amssymb}
\usepackage{array}
\usepackage{mathrsfs}

\title{Homework 4}
\author{Aaron Ma}
\begin{document}
\maketitle

\section{1}
\subsection{14}
Consider the statement: The cube of any rational number is a rational number.\\
Write the statement formally using a quantifier and a variable.\\
Determine whether the statement is true or false and justify your answer.

\subsubsection{1}
\begin{align*}
    \forall x(x\in \mathbb{Q} \rightarrow x^3 \in \mathbb{Q})
\end{align*}
\subsubsection{2}
\begin{align*}
    \intertext{Proof:}\\
    \intertext{Let $x \in \mathbb{Q}$}
    \intertext{By definition of rational numbers, there exists some}
    p,q \in \mathbb{Z}(x = \frac{p}{q}) \\
    \therefore x^3 &= (\frac{p}{q})^3 \tag{By substitution}\\
    &= \frac{p \cdot p \cdot p}{q \cdot q \cdot q}\tag{By algebra}\\
    \intertext{As the integer set is closed under multiplication, there exists}
    s = p \cdot p \cdot p \in \mathbb{Z}, r = q \cdot q \cdot q \in \mathbb{Z}\\
    \therefore x^3 = \frac{s}{r}\tag{By substitution}\\
    \intertext{As $s,r \in \mathbb{Z}, x^3 \in \mathbb{Q}$}
\end{align*}

\subsection{18}
The statement is True.\\
If $r$  and $s$ are any two rational numbers, then $\frac{r + s}{2}$ is rational.\\\\
Proof:
\begin{align*}
    \intertext{Let $r,s \in \mathbb{Q}$, then by the definition of rational numbers, there exists some $a,b,c,d \in \mathbb{Z}$ so that $r = \frac{a}{b}, s = \frac{c}{d}$ }
    \therefore \frac{r+s}{2} &= \frac{\frac{a}{b} + \frac{c}{d}}{2}\tag{By substitution}\\
    &= \frac{ad + cb}{2bd}\tag{By algebra}\\
    \intertext{As integer set is closed under addition and multiplication}\\
    (ad+cb) \in \mathbb{Z}, 2bd \in \mathbb{Z}\\
    \intertext{By definition of rational numbers} \frac{r + s}{2} \in \mathbb{Q}
\end{align*}


\section{2}
\subsection{32}
For every real number $c$, if $c$ is a root of a polynomial with rational coefficients, then $c$ is a root of a polynomial with integer coefficients.\\
Proof:
\begin{align*}
    \intertext{Let $c$ be the root of}
    p(x) &= \sum_{k=0}^{n}m_kx^k\\
    \intertext{For some $m_k \in \mathbb{Q}$}
    \therefore
    p(x) &= \sum_{k=0}^{n}\frac{a_k}{b_k}x^k\tag{By definition of rational numbers}\\
    &= \frac{a_0}{b_0}x^0 + \frac{a_1}{b_1}x^1 + \ldots + \frac{a_n}{b_n}x^n
    \intertext{where $a_n,b_n \in \mathbb{Z}$\\
    By definition of root,} p(c) = 0\\
    \intertext{By substitution,}
    \frac{a_0}{b_0}c^0 + \frac{a_1}{b_1}c^1 + \ldots + \frac{a_n}{b_n}c^n = 0\\
    \intertext{Let} 
    q(x) &= \prod_{k=1}^nb_kp(x)\\
    &=  {a_0}x^0 + {a_1}x^1 + \ldots + {a_n}x^n\tag{By Algebra}\\
    \therefore q(c) &= {a_0}c^0 + {a_1}c^1 + \ldots + {a_n}c^n \tag{By substitution}\\
    &= \prod_{k=1}^nb_kp(c) \tag{By substitution}\\
    &= 0 \tag{By algebra}\\
    \intertext{By definition of root, $c$ is the root for $q(x)$}
    \intertext{As $a_0,a_1, \ldots, a_n \in \mathbb{Z}$\\
    q(x) is a polynomial with integer coefficient
    In conclusion, $c$ is a root of a polynomial with integer coefficients}\\
\end{align*}


\subsection{33}
\subsubsection{a}
\begin{align*}
    \intertext{Let $(x-r)(x-s) = kx^2 - a x + b$}
    (x-r)(x-s) &= x^2 - (r+s)x + rs\\
    &= x^2 - ax + b\\
    \therefore 
    a = r+s, b = rs, k = 1 \in \mathbb{Z} - 2 \mathbb{Z}\\
    \intertext{Situation 1:}
    \intertext{if $r,s \in 2\mathbb{Z}$}\\
    r = 2p, s = 2q \intertext{for some $p,q \in \mathbb{Z}$}
    \therefore
    a = r+s &= 2p + 2q \tag{By substitution}\\
    &= 2(p+q)\tag{By algebra}\\
    \intertext{As $\mathbb{Z}$ is closed under addition}
    \intertext{(a+b) is also integer and $r+s \in 2\mathbb{Z}$}
    b &= rs \\
    &= (2p)(2q) \tag{By substitution}\\
    &= 2(2pq) \tag{By algebra}\\
    \intertext{As $\mathbb{Z}$ is closed under multiplication, $2ab \in \mathbb{Z}$}
    \intertext{Therefore,}
    a,b \in 2\mathbb{Z}\\
    \intertext{Situation 2:}
    \intertext{if $r,s \in \mathbb{Z} - 2\mathbb{Z}$}\\
    r = 2p + 1, s = 2q + 1 \intertext{for some $p,q \in \mathbb{Z}$}
    \therefore
    a = r+s &= 2p + 1 + 2q + 1 \tag{By substitution}\\
    &= 2(p+q + 1)\tag{By algebra}\\
    \intertext{As $\mathbb{Z}$ is closed under addition}
    \intertext{$(a + b + 1)$ is also integer and $r+s+1 \in 2\mathbb{Z}$}
    b &= rs \\
    &= (2p + 1)(2q + 1) \tag{By substitution}\\
    &= 2(2pq + p + q) + 1 \tag{By algebra}\\
    \intertext{As $\mathbb{Z}$ is closed under multiplication and addition, $(2pq + p + q) \in \mathbb{Z}$}
    \intertext{Therefore,}
    a \in 2\mathbb{Z}, b \in \mathbb{Z} - 2\mathbb{Z}
    \intertext{Situation 3:}
    \intertext{if there is one odd number and one even number in $r,s$, Let $r \in 2\mathbb{Z}, r\in \mathbb{Z} - 2\mathbb{Z}$}\\
    r = 2p, s = 2q + 1 \intertext{for some $p,q \in \mathbb{Z}$}
    \therefore
    a = r+s &= 2p + 2q + 1 \tag{By substitution}\\
    &= 2(p+q) + 1\tag{By algebra}\\
    \intertext{As $\mathbb{Z}$ is closed under addition}
    \intertext{$(q + p)$ is also integer and $r+s \in 2\mathbb{Z}$}
    b &= rs \\
    &= (2p)(2q + 1) \tag{By substitution}\\
    &= 2(2pq + p) \tag{By algebra}\\
    \intertext{As $\mathbb{Z}$ is closed under multiplication and addition, $(2pq + p) \in \mathbb{Z}$}
    \intertext{Therefore,}
    a \in 2\mathbb{Z}, b \in \mathbb{Z} - 2\mathbb{Z}
\end{align*}
\subsubsection{b}
From (a), we can see that no matter whether the two integers are odd or even, there is no chance that a and b are both odd. As a result, it is impossible.

\subsection{36}
The proof chooses a specific set of r and s instead of arbitary r and s.

As a result, it cannot be generalized for all rational numbers and the proof is false
\end{document}