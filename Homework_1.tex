\documentclass{article}

\usepackage{graphicx} % Required for inserting images
\usepackage{amssymb}
\usepackage{amsmath}
\usepackage{array}
\usepackage{xcolor}

\title{Homework 1}
\author{Aaron Ma}
\begin{document}
\maketitle
- Failure to submit homework correctly will result in zeroes. Student graders cannot see your work if you do not select the pages of your work correctly.

- Handwritten homework is OK. You do not have to type up your work.

- Problems assigned from the textbook are from the $5^{th}$ edition.

- No late homework accepted. Lateness due to technical issues will not be excused.

Theorem *. Let $p$ and $q$ be statement variables. $p \rightarrow q \equiv \neg p \vee q$.

1. (6 points) Show the logical equivalences using truth tables and say a few words explaining why your truth table shows $\equiv$

(a) $\neg(p \vee q) \equiv \neg p \wedge \neg q$ (DeMorgan's law).


\begin{center}
    \begin{tabular}{c|c|c|c|c|c|c}
     $p$ & $q$ & $\neg p$ & $\neg q$ & $p \vee q $ & $\neg (p \vee q)$ & $\neg p  \wedge \neg q$\\
     \hline
     T & T & F & F & T & F & F\\
     T & F & F & T & T & F & F\\
     F & T & T & F & T & F & F\\
     F & F & T & T & F & T & T\\
    \end{tabular}

\end{center}

As seen in the table, $\neg(p \vee q)$ always has the same truth value as $\neg p \wedge \neg q$ does regardless of the truth value of $p$ and $q$, which are the only statements related to the two statement forms. As a result, the two statement results are equivalent
\\


(b) $p \vee(q \wedge r) \equiv(p \vee q) \wedge(p \vee r)$ (Distributive law).

Note that your truth table will have 8 rows.


\begin{center}
    \begin{tabular}{c|c|c|c|c|c|c|c}
     $p$ & $q$ & $r$ & $q \wedge r$ & $p \vee (q \wedge r) $ & $p \vee q$ & $p \vee r$ & $(p \vee q) \wedge ( p \vee r)$\\
     \hline
     T & T & T & T & T & T & T & T\\
     T & T & F & F & T & T & T & T\\
     T & F & T & F & T & T & T & T\\
     T & F & F & F & T & T & T & T\\
     F & T & T & T & T & T & T & T\\
     F & T & F & F & F & T & F & F\\
     F & F & T & F & F & F & T & F\\
     F & F & F & F & F & F & F & F\\
    \end{tabular}

\end{center}

As seen in the table, $p \vee(q \wedge r)$ always has the same truth value as $(p \vee q) \wedge(p \vee r)$ does regardless of the truth value of $p$, $q$, and $r$, which are the only statements related to the two statement forms. As a result, the two statement results are equivalent





2. (3 points) Prove that $(p \vee q) \rightarrow r \equiv(p \rightarrow r) \wedge(q \rightarrow r)$ using Theorem * and Theorem 2.1.1. Annotate your proof. For reference, see example 2.1.14.\\
$Proof:$
\begin{align}
    (p \vee q) \rightarrow r &\equiv \neg (p \vee q) \wedge r  \tag{by Theorem *} \\
                            &\equiv (\neg p \wedge \neg q) \wedge r \tag{by DeMorgan's Law}\\
                            &\equiv (\neg p \wedge r ) \wedge (\neg q \wedge r) \tag{by Distributive Law}\\
                            &\equiv (p \rightarrow r) \wedge (q \rightarrow r) \tag{by Theorem *} 
\end{align}


3. (3 points) Find all values of $p$ and $q$ for which $p \rightarrow q$ is not equal to $q \rightarrow p$. For which values of $p, q$ are the statement forms equal?

We can construct a truth table for this
\begin{center}
    \begin{tabular}{c|c|c|c}
        $p$ & $q$ & $p \rightarrow q$ & $q \rightarrow p$  \\
         T & T & T & T \\
         T & F & F & T \\
         F & T & T & F \\
         F & F & T & T
    \end{tabular}
\end{center}
We can find that when $p$ and $q$ are both $True$ or both $False$, the statement forms are equal, when $p$ and $q$ are not equal, the statement forms are not equal
 
4. (3 points) Show that $[(p \rightarrow q) \wedge(p \rightarrow \neg q)] \rightarrow \neg p$ is a tautology using Theorem * and Theorem 2.1.1. Annotate your proof. For reference, see example 2.1.14.\\
$Proof:$
\begin{align}
    [(p \rightarrow q) \wedge (p \rightarrow \neg q)] \rightarrow \neg p &\equiv 
    \neg [(p \rightarrow q) \wedge (p \rightarrow \neg q)] \vee \neg p \tag{by Theorem *}\\
    &\equiv \neg[(\neg p \vee q) \wedge (\neg p \vee \neg q)] \vee \neg p \tag{by Theorem *}\\
    &\equiv \neg[\neg p \vee (q \wedge \neg q)]\vee \neg p \tag{by Distributive Law}\\
    &\equiv \neg(\neg p \vee c) \vee \neg p \tag{by Negation Law}\\
    &\equiv \neg(\neg p) \vee \neg p \tag{by Universal Bound Law}\\
    &\equiv p \vee \neg p \tag{by Double Negative Law}\\
    &\equiv t \tag{by Negation Law}  
\end{align}


5. (9 points) Section 2.1 \#31.\\
$a: $ {01,02,11,12}\\
$b: $ {21,22}\\
$c: $ {10,11,20,21}

6. (3 points) Section 2.1 \#46(a).\\
for $p \oplus p$
\begin{align}
    p \oplus p &\equiv (p \vee p) \wedge \neg (p \wedge p)\\
    &\equiv p \wedge\neg p\\
    &\equiv c
\end{align}
\\Similarly, for $(p \oplus p) \oplus p$\\
\begin{align}
    (p \oplus p) \oplus p &\equiv c \oplus p\\
    &\equiv (c \vee p) \wedge \neg (c \wedge p)\\
    &\equiv p \wedge \neg c\\
    &\equiv p \wedge t\\
    &\equiv p
\end{align}


7. (24 points) Section 2.2 \#22, 23 .

Remark. For both problems, parts (b), (c), (e), (g) only.\\
\#22\\ 
(b) If tomorrow is not January, then today is not New Year's Eve\\
(c) If r is not rational, then the decimal expansion of r is not terminating\\
(e) If x is neither positive nor 0, then x is not nonnegative\\
(g) If n is not divisible by 2 or n is not divisible by 3, then n is not divisible by 6\\
\#23\\
Converse:\\
(b)If tomorrow is January, then today is New Year's Eve\\
(c)If r is rational, then the decimal expansion of r is terminating\\
(e)If x is positive or x is 0, then x is nonnegative\\
(g)If n is divisible by 2 and n is divisible by 3, then n is divisible by 6\\
Inverse:\\
(b)If today is not New Year's Eve, then tomorrow is not January\\
(c)If the decimal expansion of r is not terminating, then r is not rational\\
(e)If x is nonnegative, then x is neither positive nor 0\\
(g)If n is not divisible by 6, then n is not divisible by 2 or n is not divisible by 3\\
8. (9 points) Section $2.2 \# 14,38,43$.

Remark. When doing \#14, do not use truth tables. Use Theorem * and Theorem 2.1.1.\\
\#14\\
a
\begin{align}
   p \rightarrow q \vee r &\equiv  \neg p \vee (q \vee r)\\
   &\equiv (\neg p \vee q) \vee r\\
   &\equiv \neg(p \wedge \neg q) \vee r\\
   &\equiv (p \wedge \neg q) \rightarrow r
\end{align}\\
Similarly,
\begin{align}
   p \rightarrow q \vee r &\equiv  \neg p \vee (q \vee r)\\
   &\equiv \neg p \vee (r \vee q)\\
   &\equiv (\neg p \vee r) \vee q\\ 
   &\equiv \neg(p \wedge \neg r) \vee q\\
   &\equiv (p \wedge \neg r) \rightarrow q
\end{align}\\
b\\
$1: $ If n is prime and n is not odd, then n is 2\\
$2: $ If n is prime and n is not 2, then n is odd\\
\#38\\
If it doesn't rain, Ann will not go\\
\#43\\
If Jim passes the exam, he does the homework regularly

9. (9 points) Section $2.3 \# 9,12(b), 23$.\\
Remark. In addition to the instructions associated to these problems, complete all truth tables even if the argument is invalid. Identify all critical rows even if the argument is invalid.\\
\#9\\
\begin{center}
    \begin{tabular}{|c|c|c|c|c|c|c|c|c|c|}
\hline
\multicolumn{3}{|c|}{} & \multicolumn{3}{|c|}{} & \multicolumn{3}{|c|}{Premises} & \multicolumn{1}{|c|}{Conclusion} \\
\hline
$p$ & $q$ & $r$ & $p \wedge q$ & $\neg r$ & $\neg q$ & $p \wedge q \rightarrow \neg r $&$ p \vee \neg q$ & $\neg q \rightarrow p$ & $\neg r$\\

\hline
T & T & T & T & F & F & F & T & T &  \\
T & T & F & T & T & F & T & T & T & T \\
T & F & T & F & F & T & T & T & T & T \\
T & F & F & F & T & T & T & T & T & T \\
F & T & T & F & F & F & F & F & T &  \\
F & T & F & F & T & F & T & F & T &  \\
F & F & T & F & F & T & T & T & F &  \\
F & F & F & F & T & T & T & T & F & \\
\hline
\end{tabular}
\end{center}
For all the situations where all the premises are $True$, the conclusion is $True$.\\ 
As a result, The argument is valid.

\#12(b)\\
\begin{center}
    \begin{tabular}{|c|c|c|c|c|}
    \hline
    \multicolumn{2}{|c|}{} &\multicolumn{2}{|c|}{Premises} &\multicolumn{1}{|c|}{Conclusion}\\
    \hline
        $p$ & $q$ & $p \rightarrow q$ & $\neg p$ & $\neg q$ \\
    \hline
        T & T & T & F &    \\
        T & F & F & T &    \\
        F & T & T & T & F  \\
        F & F & T & T & T  \\
    \hline
    \end{tabular}
\end{center}
We notice that in the third row, while both premises are $True$, the Conclusion is $False$.\\
This suggests that the argument is invalid.

\#23\\
\begin{align*}
     &\text{Let } p \text{ represent "Oleg is a math major",} \\
    &\text{let } q \text{ represent "Oleg is an economics major", and} \\
    &\text{let } r \text{ represent "Oleg is required to take Math 362".}
\end{align*}
Then the argument has the form\\
\begin{center}
$\neg p \rightarrow q$\\
$p \rightarrow r$\\
$\therefore q \vee \neg r$

    \begin{tabular}{|c|c|c|c|c|c|c|c|}
    \hline
    \multicolumn{5}{|c|}{} & \multicolumn{2}{|c|}{Premises} & \multicolumn{1}{|c|}{Conclusion}\\
    \hline
        $p$ & $q$ & $r$ & $\neg p$ & $\neg r$ & $\neg p \rightarrow q$ & $p 
        \rightarrow r$& $ q \vee \neg r $ \\
    \hline
        T & T & T & F & F & T & T & T \\
        T & T & F & F & T & T & F &   \\
        T & F & T & F & F & T & T & T \\
        T & F & F & F & T & T & F &   \\
        F & T & T & T & F & T & T & T \\
        F & T & F & T & T & T & T & T \\
        F & F & T & T & F & F & T &   \\
        F & F & F & T & T & F & T &   \\ 
    \hline
    \end{tabular}
\end{center}
As we can see from the truth table, for all the circumstances that both of the Premises are $True$, the Conclusion is also $True$.\\
This indicates that the argument is valid.



10. (6 points) Section $2.3 \# 29,38(d)$.
\#29\\
If at least one of these two numbers is divisible by 6, then the product of these two numbers is divisible by 6.

Neither of these two numbers is divisible by 6.
 
The product of these two numbers is not divisible by 6.\\
Let $p$ mean "At least one of these two numbers is divisible by 6"
and $q$ mean "The product of these two numbers is divisible by 6"\\
Then the argument can be changed into:\\
\begin{align*}
    p \rightarrow q \\
    \neg p\\
    \therefore \neg q \tag{Invalid: inverse error}
\end{align*}

\#38(d)\\
Consider U's words:
None of us is a knight.\\
If U is a knight, i.e., His word is true.\\
Then his word contradicts with his identity, which is knight.\\
As a result, he must be a knave\\
Next, Consider V's words: At least three of us are knights.\\
If V's a knight, then his words are true.\\
Therefore, there are at least 2 knights among W, X,Y, and Z\\
As X, Y, and Z are exclusive from each other, there is at most 1 knight among them, which means W is also a knight. However, that situation is also impossible as V and W are both knights means that there are 3 knights in total, which contradicts any of X, Y, and Z's words.\\
Therefore, V is a knave, which means X is a knave and W is a knight.\\
We now know that there is either one knight or two knights according to V's words, which corresponds to Z's and Y's words respectively. \\
If there is only one knight, then Z's word is correct, meaning he is a knight. However, as W is also a knight, it contradicts the assumption that there is only one knight.\\
As a result, there are two knights, which are W and Y.






11. (6 points) Section 2.3 \#42, 44. Annotate.


\end{document}